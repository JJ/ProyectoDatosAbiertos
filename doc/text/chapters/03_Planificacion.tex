\chapter{Planificación inicial del Trabajo}

\section{Modelo de desarrollo}

Si bien no existe una metodología correcta para el desarrollo de ontologías, principalmente debido a que un punto muy influyente la percepción que tengamos nosotros mismos de la "realidad" que queremos modelar, si que existen propuestas para llevar a cabo esta tarea.

\subsection{METHONTOLOGY}

Esta metodología divide el desarrollo de una ontología en las siguientes fases:

\begin{itemize}
	\item \textbf{Construir el glosario de términos}: incluimos todos los términos relevantes del dominio (conceptos, instancias, atributos, relaciones) y sus descripciones en lenguaje natural.
	\item \textbf{Construir taxonomías de conceptos}: seleccionamos todos los conceptos que hemos definido en nuestro glosario y definimos con la taxonomía la jerarquía en estos conceptos.
	\item \textbf{Construir un diagrama de relaciones binarias}: consiste en establecer las relaciones entre los conceptos de la ontología y sus inversas.
	\item \textbf{Construir el diccionario de conceptos}: especificar qué propiedades describen cada concepto de la taxonomía, las relaciones del diagrama y las instancias de cada uno de los conceptos.
	\item \textbf{Describir en detalle las relaciones binarias}: especificando nombre, origen, destino, cardinalidad y relación inversa.
	\item \textbf{Describir en detalle los atributos de instancia}: especificando nombre, concepto al que pertenece, tipo de valor, rango de valores, cardinalidad.
	\item \textbf{Describir en detalle los atributos de clase}: especificando nombre, concepto donde se define, tipo de valor, rango de valores, cardinalidad.
	\item \textbf{Describir en detalle las constantes}: especificando nombre, tipo de valor, valor y unidad de medida (si es numérica).
	\item \textbf{Definir los axiomas formales}: especificando nombre, descripción de la restricción, expresión lógica a cumplir, conceptos a los que afecta, relaciones a las que afecta y variables que usa.
	\item \textbf{Definir reglas}: especificando nombre, descripción de la regla, expresión en forma si-entonces, conceptos a los que afecta, atributos a los que afecta, relaciones a las que afecta y variables que usa.
	\item \textbf{Describir instancias}: especificando nombre de la instancia, nombre del concepto al que pertenece y los valores de los atributos si se conocen.
\end{itemize}

\subsection{On-To-Knowledge}

Esta metodología divide el desarrollo de una ontología en las siguientes fases:

\begin{itemize}
	\item \textbf{Estudio de viabilidad}: se identifica el problema a resolver y sus posibles solucions.
	\item \textbf{Arranque}: se crea un documento de especificaciones de requisitos de la ontología: objetivo, dominio, alcance, aplicaciones que van a hacer uso de ella, fuentes de conocimiento, usuarios y escenarios, preguntas de competencia y ontologías existentes.
	\item \textbf{Refinamiento}: se realiza una clasificación de los términos de la ontología en vista de una posterior formalización con relaciones y axiomas.
	\item \textbf{Evaluación}: se comprueba la utilidad de ontología poniendo a prueba que cumple con los requisitos definidos en el documento de especificaciones.
	\item \textbf{Mantenimiento}: se actualiza la ontología en función de los cambios que se haga en las especificaciones durante el ciclo de vida de la propia ontología.
\end{itemize}

\subsection{Protégé}

Protégé es un editor de ontologías de código abierto desarrollado por la Universidad de Stanford que permite desarrollar ontologías siguiendo una metología creada por los propios desarrolladores de Protégé. Se componente de las siguientes fases:

\begin{itemize}
	\item \textbf{Determinar el dominio y el alcance de la ontología}: se establecen cuestiones como para qué se va a utilizar, qué preguntas ha de responder o a quién va dirigida.
	\item \textbf{Considerar la reutilización de ontologías existentes}: se busca reaprovechar recursos de ontologías existentes que se pueden considerar de utilidad para nuestro caso; además, de facilitar la interconexión con otras aplicaciones que hagan uso de otro ontologías.
	\item \textbf{Enumerar términos importantes para la ontología}: realizando una lista de conceptos que se quieren tratar con lo ontología.
	\item \textbf{Definir las clases y su jerarquía}: usando los conceptos de la lista anterior para dar forma a la taxonomía de la ontología.
	\item \textbf{Definir las propiedades de las clases}: para así darle capacidad de representación de información de las clases.
	\item \textbf{Definir las restricciones de las propiedades}: tipo de valores, cardinalidad, dominio, rango...
	\item \textbf{Crear instancias}: indicando los valores de las propiedades de cada una de las instancias de las clases.
\end{itemize}
	
\subsection{Metodología seleccionada}

Debido a que en el problema que abarca este Trabajo no se parte de un sistema inexistente, si no que se va a trabajar sobre unos conceptos ya existentes, el uso de metodologías tan formales como METHONTOLOGY o On-To-Knowledge se hace algo de difícil de adaptación, es por eso que se han incluido como simples planteamientos teóricos.

\bigskip
En el caso de la ontología usada en Protégé, aunque existe el mismo problema, si es cierto que se puede adaptar a nuestro caso, por lo que es lo que se va a utilizar en el desarollo.

\section{Gestión de recursos}

\subsection{Personal}

\subsection{Hardware}

\subsection{Software}

\section{Planificación temporal}
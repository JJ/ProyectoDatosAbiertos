\chapter{Análisis y Diseño}

\section{Ontologías}
Aunque en la introducción hemos explicado brevemente lo que era una ontología, lo que básicamente describíamos como una definición formal de los conceptos del dominio de interés con el que estamos tratando de indicar cuáles son sus tipos, propiedades y relaciones para conocer el estado completo de nuestro ``mundo". Pues para construir estas ontologías podremos usar dos lenguajes independientes, pero que que se suelen usar conjuntamente: RDF Schema y OWL.

\subsection{RDF Schema}
RDF Schema (o simplemente RDFS) es una extensión del RDF original que utiliza su misma estructura y que aparece para solucionar problemas básicos en la definición de ontologías como pueden ser la declaración de clases, las restricciones entre ellas y las restricciones de sus propiedades; ya que en la versión original de RDF solo eran posible definir tipos (lo que no es exactamente igual que una clase) y las propiedades de los mismos.

\bigskip
Gracias a que RDFs nos provee de los elementos básicos y comunes para la descripción de los datos de nuestro dominio, esto nos puede servir para reutilizar conceptos que son comunes en varios dominios distintos. Para conseguir esto RDFS nos permite crear esquemas sencillos usando clases y subclases, además de permitirnos definir propiedades, dominios de aplicación y rangos de valores.

\newpage
\begin{itemize}
	\item Las clases son conjuntos de recursos que tienen características comunes y una representación en el mundo real. Tenemos tres clases fundamentales en RDFs a partir de las que podemos definir nuestras propias clases, siempre posible crear además una jerarquía de clases.
	\begin{itemize}
		\item rdfs:Class es la clase que utilizaremos para definir otras clases.
		\item rdfs:Property es la clase a partir de la cual definiremos nuevas propiedades que nos permitan describir nuestros recursos.
		\item rdfs:Resource todas las cosas descritas en RDF son llamadas ``recursos" y son instancias de esta clase, es por eso que esta clase nos permite referenciar cualquier clases desde otra clase.
	\end{itemize}
	
	\item Las propiedades nos permitirán describir las características que tienen los distintos recursos, además de las propias relaciones que existen entre distintos recursos.
	\begin{itemize}
		\item rdf:type nos permite definir el tipo de un determinado recurso, además, un recursos puede ser una instancia de más de una clase.
		\item rdfs:subClassOf es la que nos permitirá definir una jerarquía de clases y subclases, además, una clase puede ser subclase de otras subclases.
		\item rdfs:subPropertyOf, al igual que podemos crear subclases, también podemos crear subpropiedades para establecer una jerarquía de las mismas.
	\end{itemize}
	
	\item Las restricciones son las que nos permiten definir las clases sobre la que pueden aplicarse determinadas propiedades y posibles valores:
	\begin{itemize}
		\item rdfs:domain definiría las clases sobre las que se puede aplicar una propiedad.
		\item rdfs:range definiría los valores que puede tener una propiedad.
	\end{itemize}
\end{itemize}

\subsection{OWL}

OWL (Ontology Web Language) es un lenguaje de especificación de ontologías extensión de RDF Schema, proporcionando un mayor significado y semántica a las ontologías, ya que nos permite definir restricciones sobre las propiedades (como por ejemplo, los valores que puede tomar una clase) y definir axiomas, sentencias que son siempre ciertas y que nos serán muy útiles a la hora de realizar restricciones (como por ejemplo, que un recurso no puede pertenecer a dos clases que sean disjuntas). Utilizando esta jerarquía de clases y propiedades que forman la ontología, OWL se basa en lógicas descriptivas para realizar razonamientos.

\newpage
Por otra parte, OWL se puede clasificar en tres tipos, según su nivel de expresividad:

\begin{itemize}
	\item OWL Lite. Es la versión más simple y nos provee únicamente de los elementos necesarios para definir una jerarquía de clases y propiedades con restricciones básicas como los tipos de valores, cardinalidades o propiedades de las características (inversa, transitiva, simétrica, funcional e inversa funcinal) entre otras.
	\item OWL DL. Posee mayor expresividad que la anterior, permitiendo además que las jerarquías sean razonadas automáticamente por inferencia que además permitan encontrar posibles inconsistencias mediante el uso de diferentes axiomas (como especificar clases disjuntas)
	\item OWL Full. Es la versión con un nivel mayor de expresividad, pero que está basadas en semánticas diferentes a OWL Lite y OWL DL. Por ejemplo, en OWL Full, una clase puede ser simultáneamente un colección de individuos y un individuo en sí mismo, algo que no está permitido en OWL DL.
\end{itemize}

Y después de hablar sobre los tipos de clasificación de OWL, tenemos que especificar sus componentes generales:

\begin{itemize}
	\item Clases e instancias: todas las clases (owl:Class) son a su vez subclases de una única superclase (owl:THing). Además las subclases e instancias se definen con RDFS mediante las propiedades rdfs:subClassOf y rdfs:type respectivamente. También una clase puede crearse como la intersección de varias clases (owl:intersectionOf), la unión de varias clases (owl:unionOf) o el complemento de otra clase (owl:complementOf).
	\item Propiedades: podemos definir propiedades literales de una clase con owl:DatatypeProperty, indicando además el tipo de dato de XML Schema del que es dicha propiedad; además, también podemos indicar las propiedades que relacionan distintas clases con owl:ObjectProperty.
	\item Características de las propiedades: OWL nos permite definir diferentes tipos de propiedades:
	\begin{itemize}
		\item Transitiva (owl:TransitiveProperty): si el sujeto A está relacionado con el sujeto B por medio de la propiedad P y por otro lado el sujeto B está relacionado con el sujeto C también por medio de la propiedad P, si la propiedad P está definida como una propiedad transitiva, se puede inferir que el sujeto A está relacionado con el sujeto C por medio de la propiedad P.
		\item Simétrica (owl:SymmetricProperty): si el sujeto A está relacionado con el sujeto B por medio de la propiedad P, si la propiedad P está definida como una propiedad simétrica, se puede inferir que el sujeto B está relacionado con el sujeto A por medio de la propiedad P.
		\item Funcional (owl:FunctionalProperty): si el sujeto A está relacionado con el sujeto B por medio de la propiedad P y por otro lado el sujeto B está relacionado con el sujeto C también por medio de la propiedad P, si la propiedad P está definida como una propiedad funcional, se puede inferir que el sujeto B y el sujeto C son el mismo sujeto.
		\item Inversa (owl:InverseOf): si la propiedad P1 está definida como la propiedad inversa de la propiedad P2, entonces si el sujeto A está relacionado con el sujeto B por medio de la propiedad P1, se puede inferir que el sujeto B está relacionado con el sujeto A por medio de la propiedad P2.
		\item Inversa funcional (owl:InverseFunctionalProperty): si la propiedad P está definida como propiedad inversa funcional, entonces si el sujeto A está relacionado con el sujeto B por medio de la propiedad P, se puede inferir que el sujeto A y el sujeto B son en realidad el mismo sujeto.
	\end{itemize}
	\item Restricciones sobre las propiedades: una vez definidas las clases y las propiedades solo nos queda ver como se definen las restricciones sobre las propiedades. Además del dominio y el rango que ya podíamos definir con RDFS, OWL nos permite hacer restricciones más descriptivas como indicar de qué forma las instancias de una clase pueden tomar como únicos valores las instancias de otra clase (owl:allValuesFrom, owl:someValuesFrom, owl:hasValue) o la cardinalidad de los mismos (owl:maxCardinality, owl:minCardinality, owl:cardinality).
	\item Axiomas: un axioma es una sentencia que es siempre cierta, OWL DL se basa en la lógica descriptiva para partiendo de la jerarquía y restricciones de clases poder inferir de forma automática razonamientos sobre los datos que tiene. En concreto, podemos definir desde controlar la integridad de nuestros datos mediante la definición de clases o propiedades disjuntas (owl:disjointWith, owl:AllDisjointClasses, owl:AllDisjointProperties); o indicar que nuestras clases y propiedades son equivalentes a clases y propiedades de otros vocabularios (owl:equivalentClass, owl:equivalentProperty). Esto último es lo que realmente le da utilidad a tener estructuras con un marco común, ya que es lo que permite la interoperabilidad, permiten que existan los datos enlazados.
\end{itemize}

\subsection{Razonamiento e inferencia}

``Se entiende por razonamiento a la facultad que permite resolver problemas, extraer conclusiones y aprender de manera consciente de los hechos, estableciendo conexiones causales y lógicas necesarias entre ellos."\footnote{https://es.wikipedia.org/wiki/Razonamiento}.

\bigskip
Hemos dicho que el principal objetivo es conseguir que los datos sean entendibles por las máquinas, que puedan razonar sobre ellos y sacar conclusiones; precisamente para eso nos basaremos en la lógica. La lógica estudia las condiciones bajo las cuales siguiendo una serie de pasos elementales se puede pasar de una serie de premisas a una conclusión, cuando estamos tratando con máquinas el problema que encontramos en su forma de procesar la lógica es que no dispone de los mecanismos adecuados para conocer cómo y en qué orden deben realizarse esos pasos elementales, algo que si es capaz de hacer la mente humana de forma natural. 

\bigskip
Esto es lo que intenta solucionar la inteligencia artificial, y en nuestro caso concreto es lo que queremos conseguir, que a través de las ontologías definidas se puedan realizar inferencias sobre la información proporcionada por la ontología descrita, que podamos obtener resultados producto del razonamiento que lleven a que la máquina saque sus propias conclusiones.

\bigskip
Por ejemplo, si tenemos varias clases que entre sus datos cuentan con la propiedad ``hombres" y la propiedad ``mujeres" en referencia al número de hombres o mujeres que se han matriculado en una determinada titulación, proceden de un mismo país o cualquier caso similar, nuestro razonamiento humano nos hace saber que los ``hombres" y las ``mujeres" son ``personas", por lo que si quisiéramos saber cuántas ``personas" hay matriculadas en una titulación, lo que estaríamos haciendo es sumar el número de ``hombres" y de ``mujeres"; sin embargo, este razonamiento no es obvio para una máquina, necesita una regla que le diga que los ``hombres" y las ``mujeres" son ``personas".

\bigskip
Aquí es donde vuelve a entrar en juego RDFS, ya que como hemos indicado antes, su vocabulario nos permite establecer jerarquías, por lo tanto podríamos especificar una jerarquía en la que tenemos una propiedad que es ``persona" y luego indicar que ``hombres" y ``mujeres" son subpropiedades de ``persona". Gracias a esta propiedad que acabamos de definir, si la máquina quiere contar número de ``personas", pero estas no están definidas explícitamente como tales, podrá inferir que lo que debe hacer es contar los ``hombres" y las ``mujeres".

\section{Conceptos del sistema}

Los conjuntos de datos con los que se ha decidido trabajar son los siguientes:

\begin{itemize}
	\item ``Demanda académica: procedimientos acceso": contiene información relativa al número total de solicitudes de matrícula demandadas en la universidad. Datos desde el curso 2012/2013 hasta el curso 2014/2015.
	\item ``Demanda académica: titulaciones": contiene información sobre la demanda de matrícula en relación con las plazas ofertadas en titulaciones oficiales de grado en la universidad. Datos desde el curso 2014/2015 hasta el curso 2014/2015.
	\item ``Matrículas: grado": contiene información relacionada con las matrículas de titulaciones de grado realizadas en la universidad, agrupándola por rama de conocimiento, titulación y sexo del estudiante. Datos desde el curso 2010/2011 hasta el curso 2014/2015.
	\item ``Matrículas: posgrado": contiene información relacionada con las matrículas de titulaciones de posgrado realizadas en la universidad, agrupándola por rama de conocimiento, titulación y sexo del estudiante. Datos desde el curso 2010/2011 hasta el curso 2014/2015.
	\item ``Número medio de créditos": contiene información relacionada con el número medio de créditos de los estudiantes de la universidad, agrupándola por rama de conocimiento, plan de estudios, número medio de créditos matriculados, número medio de créditos presentados y número medio de créditos superados. Datos desde el curso 2012/2013 hasta el curso 2013/2014.
	\item ``Oferta de titulaciones: doctorado": contiene información relativa a la oferta de titulaciones para estudios de doctorado, agrupándola por titulación, rama de conocimiento, centro y campus. Datos desde el curso 2013/2014 hasta el curso 2015/2016.
	\item ``Oferta de titulaciones: grado": contiene información relativa a la oferta de titulaciones para estudios de grado, agrupándola por titulación, rama de conocimiento, centro y campus. Datos desde el curso 2013/2014 hasta el curso 2015/2016.
	\item ``Oferta de titulaciones: másteres oficiales": contiene información relativa a la oferta de titulaciones para estudios de másteres oficiales, agrupándola por titulación, rama de conocimiento, centro y campus. Datos desde el curso 2013/2014 hasta el curso 2015/2016.
	\item ``Origen geográfico de estudiantes por país": contiene información relativa al origen geográfico de los estudiantes de la universidad, agrupándola por país de origen y sexo del estudiante. Datos desde el curso 2013/2014 hasta el curso 2015/2016.
	\item ``Origen geográfico de estudiantes por provincia": contiene información relativa al origen geográfico de los estudiantes de la universidad, agrupándola por provincia de origen y sexo del estudiante. Datos desde el curso 2013/2014 hasta el curso 2015/2016.
	\item ``Tasas académicas por titulaciones": contiene información relativa a la tasa de rendimiento, tasa de éxito, tasa de abandono inicial, tasa de eficiencia, tasa de graduación y tasa de abandono de los estudiantes según la titulación que estén estudiando en la universidad, agrupándola por titulación, tasa de rendimiento, tasa de éxito, tasa de abandono inicial, tasa de eficiencia, tasa de graduación y tasa de abandono. Datos desde el curso 2011/2012 hasta el curso 2015/2016.
\end{itemize}

El motivo de haber seleccionado solo estos 11 conjuntos de datos de entre los 40 totales, es que dado la gran cantidad de datos que podemos encontrar en el portal, podemos encontrarnos datos de todo tipo; podemos encontrar datos de fácil interpretación como son los datos relacionados con matrículas, pero también podemos encontranos por ejemplo datos de carácter económico que utiliza una serie de códigos que nos permiten hacer un uso tan natural de la información que contienen, por eso en una primera aproximación solo vamos a incluir datos de los que como personas podríamos sacar información válida a simple vista. Además, si no se tienen datos de varios años tampoco se considerará un conjunto de datos interesante ya que no se puede comprobar si esos datos representan un hecho anecdótico o una tendencia.

\bigskip

Aclarado este punto, lo siguiente es concretar que cada uno de estos conjuntos de datos se corresponderá con una clase en nuestra ontología.

\begin{itemize}
	\item ``Demanda académica: procedimientos acceso"  $\rightarrow$  ``DemandaAcademicaAcceso"
	\item ``Demanda académica: titulaciones"  $\rightarrow$  ``DemandaAcademicaTitulacion"
	\item ``Matrículas: grado"  $\rightarrow$  ``MatriculasGrado"
	\item ``Matrículas: posgrado"  $\rightarrow$  ``MatriculasPosgrado"
	\item ``Número medio de créditos"  $\rightarrow$  ``NumMedioCreditos"
	\item ``Oferta de titulaciones: doctorado"  $\rightarrow$  ``OfertaTitulacionDoctor"
	\item ``Oferta de titulaciones: grado"  $\rightarrow$  ``OfertaTitulacionGrado"
	\item ``Oferta de titulaciones: másteres oficiales"  $\rightarrow$  ``OfertaTitulacionMaster"
	\item ``Origen geográfico de estudiantes por país"  $\rightarrow$  ``OrigenPais"
	\item ``Origen geográfico de estudiantes por provincia"  $\rightarrow$  ``OrigenProvincia"
	\item ``Tasas académicas por titulaciones"  $\rightarrow$  ``TasasAcademicasTitulacion"
\end{itemize}

\subsection{Definición de atributos, relaciones y restricciones}

Con las clases definidas, lo siguiente sería definir sus atributos, las relaciones y restricciones para tener la estructura completa de nuestro sistema. Para esto se usan básicamente dos tipos de propiedades diferentes definidas en OWL, ambas subclases de la clase rdf:Property definida en el estándar RDF:

\begin{itemize}
	\item Por un lado los atributos los definiremos con owl:DatatypeProperty, que es la propiedad que nos permite vincular valores del tipo que definamos con instancias de las clases que hemos definido.
	\item Por el otro lado las relaciones las definiremos con owl:ObjectProperty, que es la propiedad que nos permite vincular instancias de las distintas clases que hemos definido.
\end{itemize}
 
En lo que se refiere a los atributos, debemos tener en cuenta que hay atributos que se repiten en varias clases, por lo que tendremos que especificarlo en los dominios de dichas propiedades.

\begin{itemize}
		\item Campus: representa cada uno de los campus universitarios en los que está dividida la universidad.
		\item Centro: representa cada uno de los centros de la universidad, ya sean facultades, escuelas o centros adscritos.
		\item Cupo general: representa el número de matrículas realizadas por el grupo de estudiantes que pertenece al cupo general de estudiantes.
		\item Curso: representa el curso académico al que pertenecen los datos.
		\item Deportistas: representa el número de matrículas realizadas por el grupo de estudiantes que pertenece al cupo de deportistas de alto nivel o alto rendimiento.
		\item Discapacitados: representa el número de matrículas realizadas por el grupo de estudiantes que pertenece al cupo de personas con minusvalías reconocidas.
		\item Doctorado: representa las titulaciones de doctorado que se ofertan en la universidad.
		\item Estado: representa el estado de las solicitudes de matrículas demandadas en la universidad.
		\item Grado: representa las titulaciones de grado que se ofertan en la universidad.
		\item Hombres: representa el número de estudiantes del sexo masculino que estudian alguna titulación en la universidad.
		\item Máster: representa las titulaciones de másteres oficiales que se ofertan en la universidad.
		\item Mayores 25: representa el número de matrículas realizadas por el grupo de estudiantes que pertenece al cupo de personas mayores de 25 años.
		\item Mayores 40 y 45 : representa el número de matrículas realizadas por el grupo de estudiantes que pertenece al cupo de personas mayores de 40 y de 45 años.
		\item Mujeres: representa el número de estudiantes del sexo masculino que estudian alguna titulación en la universidad.
		\item Número medio de créditos matriculados:
		\item Número medio de créditos presentados:
		\item Número medio de créditos superados:
		\item País de origen: representa el número de estudiantes procedentes de un mismo país de origen que estudian alguna titulación en la universidad.
		\item Personas: representa el número de estudiantes que estudian alguna titulación en la universidad.
		\item Plan de estudios: representa el plan de estudios al que pertenece una titulación.
		\item Plazas ofertadas: representa el número de plazas ofertadas para una determinada titulación.
		\item Provincia: representa el número de estudiantes procedentes de una misma provincia del territorio español que estudian alguna titulación en la universidad.
		\item Rama: representa la rama de conocimiento a la que pertenece alguna titulación.
		\item Tasa de abandono: representa el porcentaje entre el número total de estudiantes de nuevo ingreso en una titulación que debieron obtener el título el año académico anterior y que no se han matriculado ni en ese año académico ni en el anterior.
		\item Tasa de abandono inicial: representa el porcentaje entre los estudiantes matriculados en una determinada titulación en un curso académico que no se matricularon en dicha titulación en los dos años siguientes y el número total de estudiantes que accedieron a esa misma titulación en ese mismo curso académico.
		\item Tasa de eficiencia: representa el porcentaje entre el número total de créditos de la titulación a los que deberían haberse matriculado el conjunto de estudiantes graduados en un año académico y el número total de créditos que se matricularon finalmente.
		\item Tasa de graduación: representa el porcentaje de estudiantes que finalizan una titulación en el tiempo previsto por el plan de estudios o en un año académico más y el número de estudiantes que entraron en esa misma titulación.
		\item Tasa de rendimiento: representa el porcentaje entre el número total de créditos superados (menos los créditos adaptados, convalidados y reconocidos) por los estudiantes de una titulación y el número total de créditos matriculados.
		\item Tasas académicas: representa las tasas académicas que sirven como indicadores de las diferentes titulaciones.
		\item Tipo de procedimiento: representa la forma de acceso por la que los estudiantes han realizado la solicitud de matrícula en la universidad.
		\item Titulación: representa las titulaciones de grado que se ofertan en la universidad.
		\item Titulados: representa el número de matrículas realizadas por el grupo de estudiantes que pertenece al cupo de titulados universitarios.
\end{itemize}

Con todos los atributos descritos ahora solo nos queda indicar cómo los vamos a referenciar en nuestra ontología y el tipo de dato que serán.

\begin{itemize}
	\item ``Campus" $\rightarrow$ ``campus". Tipo de dato cadena de caracteres.
	\item ``Centro" $\rightarrow$ ``centro". Tipo de dato cadena de caracteres.
	\item ``Cupo general" $\rightarrow$ ``cupoGral". Tipo de dato numérico entero no negativo.
	\item ``Curso" $\rightarrow$ ``curso". Tipo de dato cadena de caracteres.
	\item ``Deportistas" $\rightarrow$ ``deportistas". Tipo de dato numérico entero no negativo.
	\item ``Discapacitados" $\rightarrow$ ``discapacitados". Tipo de dato numérico entero no negativo.
	\item ``Doctorado" $\rightarrow$ ``doctorado". Tipo de dato cadena de caracteres.
	\item ``Estado" $\rightarrow$ ``estado". Tipo de dato cadena de caracteres.
	\item ``Grado" $\rightarrow$ ``grado". Tipo de dato cadena de caracteres.
	\item ``Hombres" $\rightarrow$ ``hombres". Tipo de dato numérico entero no negativo.
	\item ``Máster" $\rightarrow$ ``master". Tipo de dato cadena de caracteres.
	\item ``Mayores 25" $\rightarrow$ ``mayor25". Tipo de dato numérico entero no negativo.
	\item ``Mayores 40 y 45 " $\rightarrow$ ``mayor40". Tipo de dato numérico entero no negativo.
	\item ``Mujeres" $\rightarrow$ ``mujeres". Tipo de dato numérico entero no negativo.
	\item ``Número medio de créditos matriculados" $\rightarrow$ ``creditosMatriculados". Tipo de dato numérico real no negativo.
	\item ``Número medio de créditos presentados" $\rightarrow$ ``creditosPresentados". Tipo de dato numérico real no negativo.
	\item ``Número medio de créditos superados" $\rightarrow$ ``creditosSuperados". Tipo de dato numérico real no negativo.
	\item ``País de origen" $\rightarrow$ ``pais". Tipo de dato cadena de caracteres.
	\item ``Personas" $\rightarrow$ ``personas". Tipo de dato numérico entero no negativo.
	\item ``Plan de estudios" $\rightarrow$ ``planEstudios". Tipo de dato cadena de caracteres.
	\item ``Plazas ofertadas" $\rightarrow$ ``plazasOfertadas". Tipo de dato numérico entero no negativo.
	\item ``Provincia" $\rightarrow$ ``provincia". Tipo de dato cadena de caracteres.
	\item ``Rama de conocimiento" $\rightarrow$ ``ramaConocimiento". Tipo de dato cadena de caracteres.
	\item ``Tasa de abandono" $\rightarrow$ ``tasaAbandono". Tipo de dato numérico real no negativo.
	\item ``Tasa de abandono inicial" $\rightarrow$ ``tasaAbandonoInicial". Tipo de dato numérico real no negativo.
	\item ``Tasa de eficiencia" $\rightarrow$ ``tasaEficiencia". Tipo de dato numérico real no negativo.
	\item ``Tasa de graduación" $\rightarrow$ ``tasaGraduacion". Tipo de dato numérico real no negativo.
	\item ``Tasa de rendimiento" $\rightarrow$ ``tasaRendimiento". Tipo de dato numérico real no negativo.
	\item ``Tasas académicas" $\rightarrow$ ``tasasAcademicas". Tipo de dato numérico real no negativo.
	\item ``Tipo de procedimiento" $\rightarrow$ ``tipoProcedimiento". Tipo de dato numérico real no negativo.
	\item ``Titulación" $\rightarrow$ ``titulacion". Tipo de dato cadena de caracteres.
	\item ``Titulados" $\rightarrow$ ``titulados". Tipo de dato numérico entero no negativo
\end{itemize}

% Metodologías para el diseño de ontologías

\section{Diseño de la ontología}

\subsection{Vocabularios usados}

\subsection{Clases}

\begin{table}[!ht]
	\centering
	\begin{tabular}{|p{.17\textwidth}|p{.9\textwidth}|}
		\hline
		\multicolumn{2}{|l|}{Clase: \textbf{DemandaAcademicaAcceso}}
		\\ \hline
		Propiedades:&
		\begin{itemize}
			\item tipoProcedimiento
			\item estado
			\item hombres
			\item mujeres
			\item curso
		\end{itemize}
		\\ \hline
		Ejemplo:&
		\textless rdf:Description \newline
		\tab rdf:about=``/DemandaAcademicaAcceso/2012-2013\#1"\textgreater \newline
		\tab \textless rdf:type rdf:resource=``\#DemandaAcademicaAcceso"\ /\textgreater \newline
		\tab \textless ugr:tipoProcedimiento\textgreater \newline\tab\tab CONVOCATORIA ORDINARIA DE JUNIO: PRUEBA DE ACCESO A LA UNIVERSIDAD PARA ESTUDIANTES PROVENIENTES DE BACHILLERATO Y DE CICLOS FORMATIVOS DE GRADO SUPERIOR \newline\tab\textless /ugr:tipoProcedimiento\textgreater \newline
		\tab \textless ugr:estado\textgreater \newline\tab\tab PRESENTADOS FASE GENERAL\newline\tab\textless /ugr:estado\textgreater  \newline
		\tab \textless ugr:hombres rdf:datatype=``\&xsd;nonNegativeInteger"\textgreater \newline\tab\tab2264 \newline \tab \textless /ugr:hombres\textgreater \newline
		\tab \textless ugr:mujeres rdf:datatype=``\&xsd;nonNegativeInteger"\textgreater \newline\tab\tab2877 \newline \tab \textless /ugr:mujeres\textgreater  \newline
		\tab \textless ugr:curso\textgreater \newline\tab\tab2012/2013\newline\tab\textless /ugr:curso\textgreater  \newline
		\textless /rdf:Description\textgreater 
		\\ \hline
	\end{tabular}
	\caption{Clase DemandaAcademicaAcceso}
	\label{clase-demandaacademicaacceso}
\end{table}

\begin{table}[!ht]
	\centering
	\begin{tabular}{|p{.17\textwidth}|p{.9\textwidth}|}
		\hline
		\multicolumn{2}{|l|}{Clase: \textbf{DemandaAcademicaTitulacion}}
		\\ \hline
		Propiedades:&
		\begin{itemize}
			\item titulacion
			\item plazasOfertadas
			\item cupoGral
			\item mayor25
			\item mayor40
			\item titulados
			\item discapacitados
			\item deportistas
		\end{itemize}
		\\ \hline
		Ejemplo:&
		\textless rdf:Description \newline \tab rdf:about=``/DemandaAcademicaTitulacion/2014-2015\#1"\textgreater \newline
		\tab \textless rdf:type rdf:resource=``\#DemandaAcademicaTitulacion"\ /\textgreater \newline
		\tab \textless ugr:titulacion\textgreater \newline\tab\tab ADMINISTRACIÓN Y DIRECCIÓN DE EMPRESAS\newline\tab\textless /ugr:titulacion\textgreater \newline
		\tab \textless ugr:plazasOfertadas rdf:datatype=``\&xsd;nonNegativeInteger"\textgreater  \newline \tab \tab 281\newline\tab\textless /ugr:plazasOfertadas\textgreater \newline
		\tab \textless ugr:cupoGral rdf:datatype=``\&xsd;nonNegativeInteger"\textgreater \newline \tab \tab 271\newline\tab\textless /ugr:cupoGral\textgreater 
		\tab \newline \tab \textless ugr:mayor25 rdf:datatype=``\&xsd;nonNegativeInteger"\textgreater \newline \tab \tab 6\newline\tab\textless /ugr:mayor25\textgreater 
		\tab \newline \tab \textless ugr:mayor40 rdf:datatype=``\&xsd;nonNegativeInteger"\textgreater \newline \tab \tab 0\newline\tab\textless /ugr:mayor40\textgreater 
		\tab \newline \tab \textless ugr:titulados rdf:datatype=``\&xsd;nonNegativeInteger"\textgreater \newline \tab \tab 3\newline\tab\textless /ugr:titulados\textgreater 
		\tab \newline \tab 
		\textless ugr:discapacitados rdf:datatype=``\&xsd;nonNegativeInteger"\textgreater \newline \tab \tab 1\newline\tab\textless /ugr:discapacitados\textgreater 
		\tab \newline \tab \textless ugr:deportistas rdf:datatype=``\&xsd;nonNegativeInteger"\textgreater \newline \tab \tab 1\newline\tab\textless /ugr:deportistas\textgreater 
		\tab \newline \tab \textless ugr:curso\textgreater \newline\tab\tab2014/2015\newline\tab\textless /ugr:curso\textgreater \newline
		\textless /rdf:Description\textgreater 
		\\ \hline
	\end{tabular}
	\caption{Clase DemandaAcademicaTitulacion}
	\label{clase-demandaacademicatitulacion}
\end{table}

\begin{table}[!ht]
	\centering
	\begin{tabular}{|p{.17\textwidth}|p{.9\textwidth}|}
		\hline
		\multicolumn{2}{|l|}{Clase: \textbf{MatriculasGrado}}
		\\ \hline
		Propiedades:&
		\begin{itemize}
			\item ramaConocimiento
			\item titulacion
			\item hombres
			\item mujeres
			\item curso
		\end{itemize}
		\\ \hline
		Ejemplo:&
		\textless rdf:Description rdf:about=``/MatriculasGrado/2010-2011\#1"\textgreater \newline
		\tab \textless rdf:type rdf:resource=``\#MatriculasGrado"\ /\textgreater \newline
		\tab \textless ugr:ramaConocimiento\textgreater \newline \tab\tab ARTES Y HUMANIDADES\newline\tab\textless /ugr:ramaConocimiento\textgreater \newline
		\tab \textless ugr:titulacion\textgreater \newline\tab\tab GRADO EN BELLAS ARTES\newline\tab\textless /ugr:titulacion\textgreater \newline
		\tab \textless ugr:hombres rdf:datatype=``\&xsd;nonNegativeInteger"\textgreater \newline\tab\tab70\newline\tab\textless /ugr:hombres\textgreater 
		\tab \newline\tab\textless ugr:mujeres rdf:datatype=``\&xsd;nonNegativeInteger"\textgreater \newline\tab\tab159\newline\tab\textless /ugr:mujeres\textgreater 
		\tab \newline\tab\textless ugr:curso\textgreater \newline\tab\tab2010/2011\newline\tab\textless /ugr:curso\textgreater \newline
		\textless /rdf:Description\textgreater 
		\\ \hline
	\end{tabular}
	\caption{Clase MatriculasGrado}
	\label{clase-matriculasgrado}
\end{table}

\begin{table}[!ht]
	\centering
	\begin{tabular}{|p{.17\textwidth}|p{.9\textwidth}|}
		\hline
		\multicolumn{2}{|l|}{Clase: \textbf{MatriculasPosgrado}}
		\\ \hline
		Propiedades:&
		\begin{itemize}
			\item titulacion
			\item hombres
			\item mujeres
			\item curso
		\end{itemize}
		\\ \hline
		Ejemplo:&
		\textless rdf:Description \newline\tab  rdf:about=``/MatriculasPosgrado/2010-2011\#1"\textgreater \newline
		\tab \textless rdf:type rdf:resource=``\#MatriculasPosgrado"\ /\textgreater \newline
		\tab \textless ugr:titulacion\textgreater \newline\tab\tab MASTER ERASMUS MUNDUS EN EL COLOR EN LA INFORMATICA Y LA TECNOLOGIA DE LOS MEDIOS / ERASMUS MUNDUS IN COLOR IN INFORMATICS AND MEDIA TECHNOLOGY (CIMET) \newline\tab\textless /ugr:titulacion\textgreater \newline
		\tab \textless ugr:hombres rdf:datatype=``\&xsd;nonNegativeInteger"\textgreater \newline\tab\tab9\newline\tab\textless /ugr:hombres\textgreater 
		\tab \newline\tab\textless ugr:mujeres rdf:datatype=``\&xsd;nonNegativeInteger"\textgreater \newline\tab\tab9\newline\tab\textless /ugr:mujeres\textgreater 
		\tab \newline\tab\textless ugr:curso\textgreater \newline\tab\tab2010/2011\newline\tab\textless /ugr:curso\textgreater \newline
		\textless /rdf:Description\textgreater 
		\\ \hline
	\end{tabular}
	\caption{Clase MatriculasPosgrado}
	\label{clase-matriculasposgrado}
\end{table}

\begin{table}[!ht]
	\centering
	\begin{tabular}{|p{.17\textwidth}|p{.9\textwidth}|}
		\hline
		\multicolumn{2}{|l|}{Clase: \textbf{NumMedioCreditos}}
		\\ \hline
		Propiedades:&
		\begin{itemize}
			\item planEstudios
			\item ramaConocimiento
			\item creditosMatriculados
			\item creditosPresentados
			\item creditosSuperados
			\item curso
		\end{itemize}
		\\ \hline
		Ejemplo:&
		\textless rdf:Description \newline \tab rdf:about=``/NumMedioCreditos/2012-2013\#1"\textgreater \newline
		\tab \textless rdf:type rdf:resource=``\#NumMedioCreditos"\ /\textgreater 
		\tab \newline\tab\textless ugr:planEstudios\textgreater \newline\tab\tab PRIMER/SEGUNDO CICLO\newline\tab\textless /ugr:planEstudios\textgreater 
		\tab \newline\tab\textless ugr:ramaConocimiento\textgreater \newline\tab\tab ARTES Y HUMANIDADES\newline\tab\textless /ugr:ramaConocimiento\textgreater 
		\newline \tab \textless ugr:creditosMatriculados rdf:datatype=``\&xsd;decimal"\textgreater \newline\tab\tab 52.40\newline\tab\textless /ugr:creditosMatriculados\textgreater \newline
		\tab \textless ugr:creditosPresentados rdf:datatype=``\&xsd;decimal"\textgreater \newline\tab\tab 40.06\newline\tab\textless /ugr:creditosPresentados\textgreater \newline
		\tab \textless ugr:creditosSuperados rdf:datatype=``\&xsd;decimal"\textgreater \newline\tab\tab 35.08\newline\tab\textless /ugr:creditosSuperados\textgreater \newline
		\tab \textless ugr:curso\textgreater \newline\tab\tab 2012/2013\newline\tab\textless /ugr:curso\textgreater 
		\newline\textless /rdf:Description\textgreater 
		\\ \hline
	\end{tabular}
	\caption{Clase NumMedioCreditos}
	\label{clase-nummediocreditos}
\end{table}

\begin{table}[!ht]
	\centering
	\begin{tabular}{|p{.17\textwidth}|p{.9\textwidth}|}
		\hline
		\multicolumn{2}{|l|}{Clase: \textbf{OfertaTitulacionDoctor}}
		\\ \hline
		Propiedades:&
		\begin{itemize}
			\item rama
			\item titulacion
			\item campus
			\item centro
			\item curso
		\end{itemize}
		\\ \hline
		Ejemplo:&
		\textless rdf:Description \newline\tab rdf:about=``/OfertaTitulacionDoctor/2013-2014\#1"\textgreater \newline
		\tab \textless rdf:type rdf:resource=``\#OfertaTitulacionDoctor"\ /\textgreater 
		\newline \tab \textless ugr:ramaConocimiento\textgreater \newline\tab\tab ARTES Y HUMANIDADES\newline\tab\textless /ugr:ramaConocimiento\textgreater 
		\newline\tab \textless ugr:titulacion\textgreater \newline\tab\tab PROGRAMA DE DOCTORADO EN BIOMEDICINA\newline\tab \textless/ugr:titulacion\textgreater 
		\newline\tab \textless ugr:campus\textgreater \newline\tab\tab ESCUELA DE DOCTORADO\newline\tab\textless /ugr:campus\textgreater 
		\newline\tab \textless ugr:centro\textgreater \newline\tab\tab UGR\newline\tab\textless /ugr:centro\textgreater 
		\newline\tab \textless ugr:curso\textgreater \newline\tab\tab 2013/2014\newline\tab\textless /ugr:curso\textgreater 
		\newline\textless /rdf:Description\textgreater 
		\\ \hline
	\end{tabular}
	\caption{Clase OfertaTitulacionDoctor}
	\label{clase-ofertatitulaciondoctor}
\end{table}

\begin{table}[!ht]
	\centering
	\begin{tabular}{|p{.17\textwidth}|p{.9\textwidth}|}
		\hline
		\multicolumn{2}{|l|}{Clase: \textbf{OfertaTitulacionGrado}}
		\\ \hline
		Propiedades:&
		\begin{itemize}
			\item ramaConocimiento
			\item titulacion
			\item campus
			\item centro
			\item curso
		\end{itemize}
		\\ \hline
		Ejemplo:&
		\textless rdf:Description \newline\tab rdf:about=``/OfertaTitulacionGrado/2013-2014\#1"\textgreater \newline
		\tab \textless rdf:type rdf:resource=``\#OfertaTitulacionGrado"\ /\textgreater 
		\newline \tab \textless ugr:ramaConocimiento\textgreater \newline\tab\tab CIENCIAS\newline\tab\textless /ugr:ramaConocimiento\textgreater 
		\newline\tab \textless ugr:titulacion\textgreater \newline\tab\tab BIOLOGIA\newline\tab \textless/ugr:titulacion\textgreater 
		\newline\tab \textless ugr:campus\textgreater \newline\tab\tab FUENTENUEVA\newline\tab\textless /ugr:campus\textgreater 
		\newline\tab \textless ugr:centro\textgreater \newline\tab\tab FACULTAD DE CIENCIAS\newline\tab\textless /ugr:centro\textgreater 
		\newline\tab \textless ugr:curso\textgreater \newline\tab\tab 2013/2014\newline\tab\textless /ugr:curso\textgreater 
		\newline\textless /rdf:Description\textgreater 
		\\ \hline
	\end{tabular}
	\caption{Clase OfertaTitulacionGrado}
	\label{clase-ofertatitulaciongrado}
\end{table}

\begin{table}[!ht]
	\centering
	\begin{tabular}{|p{.17\textwidth}|p{.9\textwidth}|}
		\hline
		\multicolumn{2}{|l|}{Clase: \textbf{OfertaTitulacionMaster}}
		\\ \hline
		Propiedades:&
		\begin{itemize}
			\item ramaConocimiento
			\item titulacion
			\item campus
			\item centro
			\item curso
		\end{itemize}
		\\ \hline
		Ejemplo:&
		\textless rdf:Description \newline\tab rdf:about=``/OfertaTitulacionMaster/2013-2014\#1"\textgreater \newline
		\tab \textless rdf:type rdf:resource=``\#OfertaTitulacionMaster"\ /\textgreater 
		\newline \tab \textless ugr:ramaConocimiento\textgreater \newline\tab\tab  ARTES Y HUMANIDADES\newline\tab\textless /ugr:ramaConocimiento\textgreater 
		\newline\tab \textless ugr:titulacion\textgreater \newline\tab\tab MASTER UNIVERSITARIO EN ARQUEOLOGIA (M71.56.1)\newline\tab \textless/ugr:titulacion\textgreater 
		\newline\tab \textless ugr:campus\textgreater \newline\tab\tab CARTUJA\newline\tab\textless /ugr:campus\textgreater 
		\newline\tab \textless ugr:centro\textgreater \newline\tab\tab FACULTAD DE FILOSOFÍA Y LETRAS\newline\tab\textless /ugr:centro\textgreater 
		\newline\tab \textless ugr:curso\textgreater \newline\tab\tab 2013/2014\newline\tab\textless /ugr:curso\textgreater 
		\newline\textless /rdf:Description\textgreater 
		\\ \hline
	\end{tabular}
	\caption{Clase OfertaTitulacionMaster}
	\label{clase-ofertatitulacionmaster}
\end{table}

\begin{table}[!ht]
	\centering
	\begin{tabular}{|p{.17\textwidth}|p{.9\textwidth}|}
		\hline
		\multicolumn{2}{|l|}{Clase: \textbf{OrigenPais}}
		\\ \hline
		Propiedades:&
		\begin{itemize}
			\item pais
			\item hombres
			\item mujeres
			\item curso
		\end{itemize}
		\\ \hline
		Ejemplo:&
		\textless rdf:Description \newline\tab rdf:about=``/OrigenPais/2013-2014\#1"\textgreater 
		\tab \newline\tab \textless rdf:type rdf:resource=``\#OrigenPais"\ /\textgreater 
		\newline \tab \textless ugr:pais\textgreater \newline\tab\tab ALBANIA\newline\tab\textless /ugr:pais\textgreater 
		\newline\tab \textless ugr:hombres rdf:datatype=``\&xsd;nonNegativeInteger"\textgreater \newline\tab\tab 3\newline\tab\textless /ugr:hombres\textgreater 
		\newline\tab \textless ugr:mujeres rdf:datatype=``\&xsd;nonNegativeInteger"\textgreater \newline\tab\tab 2\newline\tab\textless /ugr:mujeres\textgreater 
		\newline\tab \textless ugr:curso\textgreater \newline\tab\tab 2013/2014\newline\tab\textless /ugr:curso\textgreater 
		\newline\textless /rdf:Description\textgreater 
		\\ \hline
	\end{tabular}
	\caption{Clase OrigenPais}
	\label{clase-origenpais}
\end{table}

\begin{table}[!ht]
	\centering
	\begin{tabular}{|p{.17\textwidth}|p{.9\textwidth}|}
		\hline
		\multicolumn{2}{|l|}{Clase: \textbf{OrigenProvincia}}
		\\ \hline
		Propiedades:&
		\begin{itemize}
			\item provincia
			\item hombres
			\item mujeres
			\item curso
		\end{itemize}
		\\ \hline
		Ejemplo:&
		\textless rdf:Description \newline\tab rdf:about=``/OrigenProvincia/2013-2014\#1"\textgreater 
		\tab \newline\tab \textless rdf:type rdf:resource=``\#OrigenProvincia"\ /\textgreater 
		\newline \tab \textless ugr:pais\textgreater \newline\tab\tab ALAVA\newline\tab\textless /ugr:pais\textgreater 
		\newline\tab \textless ugr:hombres rdf:datatype=``\&xsd;nonNegativeInteger"\textgreater \newline\tab\tab 24\newline\tab\textless /ugr:hombres\textgreater 
		\newline\tab \textless ugr:mujeres rdf:datatype=``\&xsd;nonNegativeInteger"\textgreater \newline\tab\tab 47\newline\tab\textless /ugr:mujeres\textgreater 
		\newline\tab \textless ugr:curso\textgreater \newline\tab\tab 2013/2014\newline\tab\textless /ugr:curso\textgreater 
		\newline\textless /rdf:Description\textgreater 
		\\ \hline
	\end{tabular}
	\caption{Clase OrigenProvincia}
	\label{clase-origenprovincia}
\end{table}

\begin{table}[!ht]
	\centering
	\begin{tabular}{|p{.17\textwidth}|p{.9\textwidth}|}
		\hline
		\multicolumn{2}{|l|}{Clase: \textbf{TasasAcademicasTitulacion}}
		\\ \hline
		Propiedades:&
		\begin{itemize}
			\item titulacion
			\item tasaRendimiento
			\item tasaExito
			\item tasaAbandonoInicial
			\item tasaEficiencia
			\item tasaGraduacion
			\item tasaAbandono
		\end{itemize}
		\\ \hline
		Ejemplo:&
		\textless rdf:Description\newline\tab rdf:about=``/TasasAcademicasTitulacion/2015-2016\#1"\textgreater 
		\newline\tab \textless rdf:type rdf:resource=``\#TasasAcademicasTitulacion"\ /\textgreater 
		\newline\tab \textless ugr:titulacion\textgreater \newline\tab\tab GRADUADO EN BIOLOGIA\newline\tab\textless /ugr:titulacion\textgreater 
		\newline\tab \textless ugr:tasaRendimiento rdf:datatype=``\&xsd;decimal"\textgreater \newline\tab\tab 74.02\newline\tab\textless /ugr:tasaRendimiento\textgreater 
		\newline\tab \textless ugr:tasaExito rdf:datatype=``\&xsd;decimal"\textgreater \newline\tab\tab 83.12\newline\tab\textless /ugr:tasaExito\textgreater 
		\newline\tab \textless ugr:tasaAbandonoInicial rdf:datatype=``\&xsd;decimal"\textgreater \newline\tab\tab 12.27\newline\tab\textless /ugr:tasaAbandonoInicial\textgreater 
		\newline\tab \textless ugr:tasaEficiencia rdf:datatype=``\&xsd;decimal"\textgreater \newline\tab\tab 98\newline\tab\textless /ugr:tasaEficiencia\textgreater 
		\newline\tab \textless ugr:tasaGraduacion rdf:datatype=``\&xsd;decimal"\textgreater \newline\tab\tab 32.73\newline\tab\textless /ugr:tasaGraduacion\textgreater 
		\newline\tab \textless ugr:tasaAbandono rdf:datatype=``\&xsd;decimal"\textgreater \newline\tab\tab30\newline\tab\textless /ugr:tasaAbandono\textgreater 
		\newline\tab \textless ugr:curso\textgreater \newline\tab\tab 2015/2016\newline\tab\textless /ugr:curso\textgreater 
		\newline\textless /rdf:Description\textgreater 
		\\ \hline
	\end{tabular}
	\caption{Clase TasasAcademicasTitulacion}
	\label{clase-tasasacademicastitulacion}
\end{table}

\subsection{Propiedades}

\begin{table}[!ht]
	\begin{center}
		\begin{tabular}{|l|l|}
			\hline
			\multicolumn{2}{|l|}{
				\textbf{\begin{tabular}[c]{@{}l@{}}
						Propiedad: \\ 
						\\
						\\
						Descripción:\end{tabular}}} \\ 
			\hline
			Dominio: &  \\ 
			\hline
			Rango: &  \\ 
			\hline
			Subpropiedad: &  \\ 
			\hline
			\multicolumn{2}{|l|}{
				{\begin{tabular}[c]{@{}l@{}}
						Esto \\
						es\\
						una\\
						prueba\end{tabular}}} \\ 
			\hline
		\end{tabular}
		\caption{My caption}
		\label{my-label}
	\end{center}
\end{table}

\begin{table}[!ht]
	\begin{center}
		\begin{tabular}{|l|l|}
			\hline
			\multicolumn{2}{|l|}{
				\textbf{\begin{tabular}[c]{@{}l@{}}
						Propiedad: \\ 
						\\
						\\
						Descripción:\end{tabular}}} \\ 
			\hline
			Dominio: &  \\ 
			\hline
			Rango: &  \\ 
			\hline
			Subpropiedad: &  \\ 
			\hline
			\multicolumn{2}{|l|}{
				{\begin{tabular}[c]{@{}l@{}}
						Esto \\
						es\\
						una\\
						prueba\end{tabular}}} \\ 
			\hline
		\end{tabular}
		\caption{My caption}
		\label{my-label}
	\end{center}
\end{table}

\begin{table}[!ht]
	\begin{center}
		\begin{tabular}{|l|l|}
			\hline
			\multicolumn{2}{|l|}{
				\textbf{\begin{tabular}[c]{@{}l@{}}
						Propiedad: \\ 
						\\
						\\
						Descripción:\end{tabular}}} \\ 
			\hline
			Dominio: &  \\ 
			\hline
			Rango: &  \\ 
			\hline
			Subpropiedad: &  \\ 
			\hline
			\multicolumn{2}{|l|}{
				{\begin{tabular}[c]{@{}l@{}}
						Esto \\
						es\\
						una\\
						prueba\end{tabular}}} \\ 
			\hline
		\end{tabular}
		\caption{My caption}
		\label{my-label}
	\end{center}
\end{table}

\begin{table}[!ht]
	\begin{center}
		\begin{tabular}{|l|l|}
			\hline
			\multicolumn{2}{|l|}{
				\textbf{\begin{tabular}[c]{@{}l@{}}
						Propiedad: \\ 
						\\
						\\
						Descripción:\end{tabular}}} \\ 
			\hline
			Dominio: &  \\ 
			\hline
			Rango: &  \\ 
			\hline
			Subpropiedad: &  \\ 
			\hline
			\multicolumn{2}{|l|}{
				{\begin{tabular}[c]{@{}l@{}}
						Esto \\
						es\\
						una\\
						prueba\end{tabular}}} \\ 
			\hline
		\end{tabular}
		\caption{My caption}
		\label{my-label}
	\end{center}
\end{table}

\begin{table}[!ht]
	\begin{center}
		\begin{tabular}{|l|l|}
			\hline
			\multicolumn{2}{|l|}{
				\textbf{\begin{tabular}[c]{@{}l@{}}
						Propiedad: \\ 
						\\
						\\
						Descripción:\end{tabular}}} \\ 
			\hline
			Dominio: &  \\ 
			\hline
			Rango: &  \\ 
			\hline
			Subpropiedad: &  \\ 
			\hline
			\multicolumn{2}{|l|}{
				{\begin{tabular}[c]{@{}l@{}}
						Esto \\
						es\\
						una\\
						prueba\end{tabular}}} \\ 
			\hline
		\end{tabular}
		\caption{My caption}
		\label{my-label}
	\end{center}
\end{table}

\begin{table}[!ht]
	\begin{center}
		\begin{tabular}{|l|l|}
			\hline
			\multicolumn{2}{|l|}{
				\textbf{\begin{tabular}[c]{@{}l@{}}
						Propiedad: \\ 
						\\
						\\
						Descripción:\end{tabular}}} \\ 
			\hline
			Dominio: &  \\ 
			\hline
			Rango: &  \\ 
			\hline
			Subpropiedad: &  \\ 
			\hline
			\multicolumn{2}{|l|}{
				{\begin{tabular}[c]{@{}l@{}}
						Esto \\
						es\\
						una\\
						prueba\end{tabular}}} \\ 
			\hline
		\end{tabular}
		\caption{My caption}
		\label{my-label}
	\end{center}
\end{table}

\begin{table}[!ht]
	\begin{center}
		\begin{tabular}{|l|l|}
			\hline
			\multicolumn{2}{|l|}{
				\textbf{\begin{tabular}[c]{@{}l@{}}
						Propiedad: \\ 
						\\
						\\
						Descripción:\end{tabular}}} \\ 
			\hline
			Dominio: &  \\ 
			\hline
			Rango: &  \\ 
			\hline
			Subpropiedad: &  \\ 
			\hline
			\multicolumn{2}{|l|}{
				{\begin{tabular}[c]{@{}l@{}}
						Esto \\
						es\\
						una\\
						prueba\end{tabular}}} \\ 
			\hline
		\end{tabular}
		\caption{My caption}
		\label{my-label}
	\end{center}
\end{table}

\begin{table}[!ht]
	\begin{center}
		\begin{tabular}{|l|l|}
			\hline
			\multicolumn{2}{|l|}{
				\textbf{\begin{tabular}[c]{@{}l@{}}
						Propiedad: \\ 
						\\
						\\
						Descripción:\end{tabular}}} \\ 
			\hline
			Dominio: &  \\ 
			\hline
			Rango: &  \\ 
			\hline
			Subpropiedad: &  \\ 
			\hline
			\multicolumn{2}{|l|}{
				{\begin{tabular}[c]{@{}l@{}}
						Esto \\
						es\\
						una\\
						prueba\end{tabular}}} \\ 
			\hline
		\end{tabular}
		\caption{My caption}
		\label{my-label}
	\end{center}
\end{table}

\begin{table}[!ht]
	\begin{center}
		\begin{tabular}{|l|l|}
			\hline
			\multicolumn{2}{|l|}{
				\textbf{\begin{tabular}[c]{@{}l@{}}
						Propiedad: \\ 
						\\
						\\
						Descripción:\end{tabular}}} \\ 
			\hline
			Dominio: &  \\ 
			\hline
			Rango: &  \\ 
			\hline
			Subpropiedad: &  \\ 
			\hline
			\multicolumn{2}{|l|}{
				{\begin{tabular}[c]{@{}l@{}}
						Esto \\
						es\\
						una\\
						prueba\end{tabular}}} \\ 
			\hline
		\end{tabular}
		\caption{My caption}
		\label{my-label}
	\end{center}
\end{table}

\begin{table}[!ht]
	\begin{center}
		\begin{tabular}{|l|l|}
			\hline
			\multicolumn{2}{|l|}{
				\textbf{\begin{tabular}[c]{@{}l@{}}
						Propiedad: \\ 
						\\
						\\
						Descripción:\end{tabular}}} \\ 
			\hline
			Dominio: &  \\ 
			\hline
			Rango: &  \\ 
			\hline
			Subpropiedad: &  \\ 
			\hline
			\multicolumn{2}{|l|}{
				{\begin{tabular}[c]{@{}l@{}}
						Esto \\
						es\\
						una\\
						prueba\end{tabular}}} \\ 
			\hline
		\end{tabular}
		\caption{My caption}
		\label{my-label}
	\end{center}
\end{table}

\begin{table}[!ht]
	\begin{center}
		\begin{tabular}{|l|l|}
			\hline
			\multicolumn{2}{|l|}{
				\textbf{\begin{tabular}[c]{@{}l@{}}
						Propiedad: \\ 
						\\
						\\
						Descripción:\end{tabular}}} \\ 
			\hline
			Dominio: &  \\ 
			\hline
			Rango: &  \\ 
			\hline
			Subpropiedad: &  \\ 
			\hline
			\multicolumn{2}{|l|}{
				{\begin{tabular}[c]{@{}l@{}}
						Esto \\
						es\\
						una\\
						prueba\end{tabular}}} \\ 
			\hline
		\end{tabular}
		\caption{My caption}
		\label{my-label}
	\end{center}
\end{table}

\begin{table}[!ht]
	\begin{center}
		\begin{tabular}{|l|l|}
			\hline
			\multicolumn{2}{|l|}{
				\textbf{\begin{tabular}[c]{@{}l@{}}
						Propiedad: \\ 
						\\
						\\
						Descripción:\end{tabular}}} \\ 
			\hline
			Dominio: &  \\ 
			\hline
			Rango: &  \\ 
			\hline
			Subpropiedad: &  \\ 
			\hline
			\multicolumn{2}{|l|}{
				{\begin{tabular}[c]{@{}l@{}}
						Esto \\
						es\\
						una\\
						prueba\end{tabular}}} \\ 
			\hline
		\end{tabular}
		\caption{My caption}
		\label{my-label}
	\end{center}
\end{table}

\begin{table}[!ht]
	\begin{center}
		\begin{tabular}{|l|l|}
			\hline
			\multicolumn{2}{|l|}{
				\textbf{\begin{tabular}[c]{@{}l@{}}
						Propiedad: \\ 
						\\
						\\
						Descripción:\end{tabular}}} \\ 
			\hline
			Dominio: &  \\ 
			\hline
			Rango: &  \\ 
			\hline
			Subpropiedad: &  \\ 
			\hline
			\multicolumn{2}{|l|}{
				{\begin{tabular}[c]{@{}l@{}}
						Esto \\
						es\\
						una\\
						prueba\end{tabular}}} \\ 
			\hline
		\end{tabular}
		\caption{My caption}
		\label{my-label}
	\end{center}
\end{table}

\begin{table}[!ht]
	\begin{center}
		\begin{tabular}{|l|l|}
			\hline
			\multicolumn{2}{|l|}{
				\textbf{\begin{tabular}[c]{@{}l@{}}
						Propiedad: \\ 
						\\
						\\
						Descripción:\end{tabular}}} \\ 
			\hline
			Dominio: &  \\ 
			\hline
			Rango: &  \\ 
			\hline
			Subpropiedad: &  \\ 
			\hline
			\multicolumn{2}{|l|}{
				{\begin{tabular}[c]{@{}l@{}}
						Esto \\
						es\\
						una\\
						prueba\end{tabular}}} \\ 
			\hline
		\end{tabular}
		\caption{My caption}
		\label{my-label}
	\end{center}
\end{table}

\begin{table}[!ht]
	\begin{center}
		\begin{tabular}{|l|l|}
			\hline
			\multicolumn{2}{|l|}{
				\textbf{\begin{tabular}[c]{@{}l@{}}
						Propiedad: \\ 
						\\
						\\
						Descripción:\end{tabular}}} \\ 
			\hline
			Dominio: &  \\ 
			\hline
			Rango: &  \\ 
			\hline
			Subpropiedad: &  \\ 
			\hline
			\multicolumn{2}{|l|}{
				{\begin{tabular}[c]{@{}l@{}}
						Esto \\
						es\\
						una\\
						prueba\end{tabular}}} \\ 
			\hline
		\end{tabular}
		\caption{My caption}
		\label{my-label}
	\end{center}
\end{table}

\begin{table}[!ht]
	\begin{center}
		\begin{tabular}{|l|l|}
			\hline
			\multicolumn{2}{|l|}{
				\textbf{\begin{tabular}[c]{@{}l@{}}
						Propiedad: \\ 
						\\
						\\
						Descripción:\end{tabular}}} \\ 
			\hline
			Dominio: &  \\ 
			\hline
			Rango: &  \\ 
			\hline
			Subpropiedad: &  \\ 
			\hline
			\multicolumn{2}{|l|}{
				{\begin{tabular}[c]{@{}l@{}}
						Esto \\
						es\\
						una\\
						prueba\end{tabular}}} \\ 
			\hline
		\end{tabular}
		\caption{My caption}
		\label{my-label}
	\end{center}
\end{table}

\begin{table}[!ht]
	\begin{center}
		\begin{tabular}{|l|l|}
			\hline
			\multicolumn{2}{|l|}{
				\textbf{\begin{tabular}[c]{@{}l@{}}
						Propiedad: \\ 
						\\
						\\
						Descripción:\end{tabular}}} \\ 
			\hline
			Dominio: &  \\ 
			\hline
			Rango: &  \\ 
			\hline
			Subpropiedad: &  \\ 
			\hline
			\multicolumn{2}{|l|}{
				{\begin{tabular}[c]{@{}l@{}}
						Esto \\
						es\\
						una\\
						prueba\end{tabular}}} \\ 
			\hline
		\end{tabular}
		\caption{My caption}
		\label{my-label}
	\end{center}
\end{table}

\begin{table}[!ht]
	\begin{center}
		\begin{tabular}{|l|l|}
			\hline
			\multicolumn{2}{|l|}{
				\textbf{\begin{tabular}[c]{@{}l@{}}
						Propiedad: \\ 
						\\
						\\
						Descripción:\end{tabular}}} \\ 
			\hline
			Dominio: &  \\ 
			\hline
			Rango: &  \\ 
			\hline
			Subpropiedad: &  \\ 
			\hline
			\multicolumn{2}{|l|}{
				{\begin{tabular}[c]{@{}l@{}}
						Esto \\
						es\\
						una\\
						prueba\end{tabular}}} \\ 
			\hline
		\end{tabular}
		\caption{My caption}
		\label{my-label}
	\end{center}
\end{table}

\begin{table}[!ht]
	\begin{center}
		\begin{tabular}{|l|l|}
			\hline
			\multicolumn{2}{|l|}{
				\textbf{\begin{tabular}[c]{@{}l@{}}
						Propiedad: \\ 
						\\
						\\
						Descripción:\end{tabular}}} \\ 
			\hline
			Dominio: &  \\ 
			\hline
			Rango: &  \\ 
			\hline
			Subpropiedad: &  \\ 
			\hline
			\multicolumn{2}{|l|}{
				{\begin{tabular}[c]{@{}l@{}}
						Esto \\
						es\\
						una\\
						prueba\end{tabular}}} \\ 
			\hline
		\end{tabular}
		\caption{My caption}
		\label{my-label}
	\end{center}
\end{table}

\begin{table}[!ht]
	\begin{center}
		\begin{tabular}{|l|l|}
			\hline
			\multicolumn{2}{|l|}{
				\textbf{\begin{tabular}[c]{@{}l@{}}
						Propiedad: \\ 
						\\
						\\
						Descripción:\end{tabular}}} \\ 
			\hline
			Dominio: &  \\ 
			\hline
			Rango: &  \\ 
			\hline
			Subpropiedad: &  \\ 
			\hline
			\multicolumn{2}{|l|}{
				{\begin{tabular}[c]{@{}l@{}}
						Esto \\
						es\\
						una\\
						prueba\end{tabular}}} \\ 
			\hline
		\end{tabular}
		\caption{My caption}
		\label{my-label}
	\end{center}
\end{table}

\begin{table}[!ht]
	\begin{center}
		\begin{tabular}{|l|l|}
			\hline
			\multicolumn{2}{|l|}{
				\textbf{\begin{tabular}[c]{@{}l@{}}
						Propiedad: \\ 
						\\
						\\
						Descripción:\end{tabular}}} \\ 
			\hline
			Dominio: &  \\ 
			\hline
			Rango: &  \\ 
			\hline
			Subpropiedad: &  \\ 
			\hline
			\multicolumn{2}{|l|}{
				{\begin{tabular}[c]{@{}l@{}}
						Esto \\
						es\\
						una\\
						prueba\end{tabular}}} \\ 
			\hline
		\end{tabular}
		\caption{My caption}
		\label{my-label}
	\end{center}
\end{table}

\begin{table}[!ht]
	\begin{center}
		\begin{tabular}{|l|l|}
			\hline
			\multicolumn{2}{|l|}{
				\textbf{\begin{tabular}[c]{@{}l@{}}
						Propiedad: \\ 
						\\
						\\
						Descripción:\end{tabular}}} \\ 
			\hline
			Dominio: &  \\ 
			\hline
			Rango: &  \\ 
			\hline
			Subpropiedad: &  \\ 
			\hline
			\multicolumn{2}{|l|}{
				{\begin{tabular}[c]{@{}l@{}}
						Esto \\
						es\\
						una\\
						prueba\end{tabular}}} \\ 
			\hline
		\end{tabular}
		\caption{My caption}
		\label{my-label}
	\end{center}
\end{table}

\begin{table}[!ht]
	\begin{center}
		\begin{tabular}{|l|l|}
			\hline
			\multicolumn{2}{|l|}{
				\textbf{\begin{tabular}[c]{@{}l@{}}
						Propiedad: \\ 
						\\
						\\
						Descripción:\end{tabular}}} \\ 
			\hline
			Dominio: &  \\ 
			\hline
			Rango: &  \\ 
			\hline
			Subpropiedad: &  \\ 
			\hline
			\multicolumn{2}{|l|}{
				{\begin{tabular}[c]{@{}l@{}}
						Esto \\
						es\\
						una\\
						prueba\end{tabular}}} \\ 
			\hline
		\end{tabular}
		\caption{My caption}
		\label{my-label}
	\end{center}
\end{table}

\begin{table}[!ht]
	\begin{center}
		\begin{tabular}{|l|l|}
			\hline
			\multicolumn{2}{|l|}{
				\textbf{\begin{tabular}[c]{@{}l@{}}
						Propiedad: \\ 
						\\
						\\
						Descripción:\end{tabular}}} \\ 
			\hline
			Dominio: &  \\ 
			\hline
			Rango: &  \\ 
			\hline
			Subpropiedad: &  \\ 
			\hline
			\multicolumn{2}{|l|}{
				{\begin{tabular}[c]{@{}l@{}}
						Esto \\
						es\\
						una\\
						prueba\end{tabular}}} \\ 
			\hline
		\end{tabular}
		\caption{My caption}
		\label{my-label}
	\end{center}
\end{table}

\begin{table}[!ht]
	\begin{center}
		\begin{tabular}{|l|l|}
			\hline
			\multicolumn{2}{|l|}{
				\textbf{\begin{tabular}[c]{@{}l@{}}
						Propiedad: \\ 
						\\
						\\
						Descripción:\end{tabular}}} \\ 
			\hline
			Dominio: &  \\ 
			\hline
			Rango: &  \\ 
			\hline
			Subpropiedad: &  \\ 
			\hline
			\multicolumn{2}{|l|}{
				{\begin{tabular}[c]{@{}l@{}}
						Esto \\
						es\\
						una\\
						prueba\end{tabular}}} \\ 
			\hline
		\end{tabular}
		\caption{My caption}
		\label{my-label}
	\end{center}
\end{table}

\begin{table}[!ht]
	\begin{center}
		\begin{tabular}{|l|l|}
			\hline
			\multicolumn{2}{|l|}{
				\textbf{\begin{tabular}[c]{@{}l@{}}
						Propiedad: \\ 
						\\
						\\
						Descripción:\end{tabular}}} \\ 
			\hline
			Dominio: &  \\ 
			\hline
			Rango: &  \\ 
			\hline
			Subpropiedad: &  \\ 
			\hline
			\multicolumn{2}{|l|}{
				{\begin{tabular}[c]{@{}l@{}}
						Esto \\
						es\\
						una\\
						prueba\end{tabular}}} \\ 
			\hline
		\end{tabular}
		\caption{My caption}
		\label{my-label}
	\end{center}
\end{table}

\begin{table}[!ht]
	\begin{center}
		\begin{tabular}{|l|l|}
			\hline
			\multicolumn{2}{|l|}{
				\textbf{\begin{tabular}[c]{@{}l@{}}
						Propiedad: \\ 
						\\
						\\
						Descripción:\end{tabular}}} \\ 
			\hline
			Dominio: &  \\ 
			\hline
			Rango: &  \\ 
			\hline
			Subpropiedad: &  \\ 
			\hline
			\multicolumn{2}{|l|}{
				{\begin{tabular}[c]{@{}l@{}}
						Esto \\
						es\\
						una\\
						prueba\end{tabular}}} \\ 
			\hline
		\end{tabular}
		\caption{My caption}
		\label{my-label}
	\end{center}
\end{table}

\begin{table}[!ht]
	\begin{center}
		\begin{tabular}{|l|l|}
			\hline
			\multicolumn{2}{|l|}{
				\textbf{\begin{tabular}[c]{@{}l@{}}
						Propiedad: \\ 
						\\
						\\
						Descripción:\end{tabular}}} \\ 
			\hline
			Dominio: &  \\ 
			\hline
			Rango: &  \\ 
			\hline
			Subpropiedad: &  \\ 
			\hline
			\multicolumn{2}{|l|}{
				{\begin{tabular}[c]{@{}l@{}}
						Esto \\
						es\\
						una\\
						prueba\end{tabular}}} \\ 
			\hline
		\end{tabular}
		\caption{My caption}
		\label{my-label}
	\end{center}
\end{table}

\begin{table}[!ht]
	\begin{center}
		\begin{tabular}{|l|l|}
			\hline
			\multicolumn{2}{|l|}{
				\textbf{\begin{tabular}[c]{@{}l@{}}
						Propiedad: \\ 
						\\
						\\
						Descripción:\end{tabular}}} \\ 
			\hline
			Dominio: &  \\ 
			\hline
			Rango: &  \\ 
			\hline
			Subpropiedad: &  \\ 
			\hline
			\multicolumn{2}{|l|}{
				{\begin{tabular}[c]{@{}l@{}}
						Esto \\
						es\\
						una\\
						prueba\end{tabular}}} \\ 
			\hline
		\end{tabular}
		\caption{My caption}
		\label{my-label}
	\end{center}
\end{table}

\begin{table}[!ht]
	\begin{center}
		\begin{tabular}{|l|l|}
			\hline
			\multicolumn{2}{|l|}{
				\textbf{\begin{tabular}[c]{@{}l@{}}
						Propiedad: \\ 
						\\
						\\
						Descripción:\end{tabular}}} \\ 
			\hline
			Dominio: &  \\ 
			\hline
			Rango: &  \\ 
			\hline
			Subpropiedad: &  \\ 
			\hline
			\multicolumn{2}{|l|}{
				{\begin{tabular}[c]{@{}l@{}}
						Esto \\
						es\\
						una\\
						prueba\end{tabular}}} \\ 
			\hline
		\end{tabular}
		\caption{My caption}
		\label{my-label}
	\end{center}
\end{table}


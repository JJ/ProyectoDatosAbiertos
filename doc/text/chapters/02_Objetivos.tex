\chapter{Objetivos}

El principal objetivo de este Trabajo Fin de Máster consiste en obtener información sobre los datos contenidos en el portal de datos abiertos de la Universidad de Granada mediante peticiones a una interfaz web. Para conseguir el resultado final esperado debemos cumplir los siguientes objetivos:

\begin{itemize}
	\item Procesamiento de los conjuntos de datos del portal de datos abiertos de la Universidad de Granada (\url{http://opendata.ugr.es/}) para convertirlos del formato CSV actual al formato triple RDF.
	\item Desarrollo de ontología que permita describir y representar la información contenida en los datos almacenados en dicho portal.
	\item Proveer de un punto de acceso a un sistema de recuperación de datos mediante SPARQL, que en este caso se montará sobre OpenLink Virtuoso, un servidor ORDBMS abierto que permite el almacenamiento y la gestión de datos en formato RDF.
\end{itemize}

Una vez finalizado todo el proceso, además de haber obtenido la posibilidad de obtener la información desde un punto de acceso, tendremos varios conjunto de datos en formato triple RDF que queremos que sean públicos al igual que los datos originales en formato CSV. Para que esto sea posible también se tratará el asunto de cómo obtener un almacén de datos operativo mediante CKAN, un sistema de gestión de datos que permite el almacenaminto y publicación de datos en distintos formatos.